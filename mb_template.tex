\documentclass[a4paper,12pt]{article}
\usepackage {parskip}
\usepackage [margin=2.5cm] {geometry}

\usepackage[utf8]{inputenc}
\usepackage[T1]{fontenc}
\usepackage{geometry}
\usepackage{amsmath, amssymb}
\usepackage{siunitx}
\usepackage{enumitem}
\usepackage{hyperref}
\usepackage{graphicx}

\usepackage{tikz}
\usepackage{tkz-euclide}

%% START SPECIAL SECTION %%
\usetkzobj{all}
\usepackage{import}
\usepackage{subcaption}
\usepackage{float}
%% END SPECIAL SECTION %%

\begin{document}

%=======%
% TITLE %
%=======%

\includegraphics[scale=0.4]{logo.png}
\begin{center}
\LARGE
\textbf{NZMOSA 2015-16 Summer Holiday Problems}
\end{center}

%==============%
% INSTRUCTIONS %
%==============%

\normalsize
\section*{Instructions}
\begin{itemize}
    \item You may attempt any question, regardless of your division.
    \item For questions 3 to 10, each question is worth 5 marks; a correct answer with no working or explanation will earn at most one of these marks. An incorrect answer with working which may lead to a full solution may earn more than one mark.
    \item No working is required for questions 1 and 2; a correct answer will earn 3 marks, and an incorrect answer will earn 0 marks.
    \item You can spend as much time on these problems as you want throughout the months of November, December and January.
    \item You may use the internet to look for terms that you don't understand. However, what you submit must represent your own individual work: you should not collaborate with anyone else for these problems.
    \item Email your solutions (scanned or typed) to \url{nzmosa@outlook.com} with your name, school and school name by the end of January.
    \item Keep in mind that we reward creative solutions!
    \item For more details, see \textit{Competitions $\to$ Competition Details} at \url{nzmosa.org}.
\end{itemize}

%==========%
% PROBLEMS %
%==========%

\section*{Problems}
\begin{enumerate}
    \item 
    Questions go here

\clearpage
\end{document}